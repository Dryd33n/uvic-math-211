\documentclass{article}
\usepackage[legalpaper, portrait, margin=0.5in]{geometry}
\usepackage{amsmath}
\usepackage{amssymb}

\title{Math 211 Ass 3.}
\author{Dryden Bryson}
\date{October 2024}

\begin{document}

\maketitle
\newpage
\section*{Question 1:}
\subsection*{a)}
We prove linearity by proving that our function satisfies \textit{additive linearity} and \textit{scalar linearity}:\\\\
\textbf{\underline{Additive Linearity:}}\\
Let $\vec{x}, \vec{y} \in \mathbb{R}^3$ then we form and simplify as follows,

\begin{table}[htp]
    \centering
    \begin{tabular}{ccll}
        $L(\vec{x}+\vec{y})$ & $=$ & $((\vec{x}+\vec{y})\times \vec{v}) + ((\vec{x}+\vec{y})\cdot \vec{v})\vec{v}$ & Definition of $L$\\
         & $=$ & $( -\vec{v}\times(\vec{x}+\vec{y})) + ( \vec{v}\cdot(\vec{x}+\vec{y}))\vec{v}$ & Anti Commutativity \& Commutativity\\
         & $=$ &$( (-\vec{v}\times\vec{x})+(-\vec{v}\times\vec{y})) + ( (\vec{v}\cdot\vec{x})+(\vec{v}\cdot\vec{y}))\vec{v}$  & Distributivity of Cross Product and Distributivity of Dot Product\\
         & $=$ & $( (-\vec{v}\times\vec{x})+(-\vec{v}\times\vec{y})) + (\vec{v}\cdot\vec{x})\vec{v}+(\vec{v}\cdot\vec{y})\vec{v}$ & Multiplicative Distributivity\\
         & $=$ & $ (-\vec{v}\times\vec{x}) + (\vec{v}\cdot\vec{x})\vec{v}+(-\vec{v}\times\vec{y})+(\vec{v}\cdot\vec{y})\vec{v}$ & Associativity \& Commutativity\\
         & $=$ & $ (\vec{x}\times\vec{v}) + (\vec{x}\cdot\vec{v})\vec{v}+(\vec{y}\times\vec{v})+(\vec{y}\cdot\vec{v})\vec{v}$ & Anti Commutativity $\times 2$ \& Commutativity $\times 2$\\
         & $=$ & $L(\vec{x})+L(\vec{y})$ & Definition of $L$\\
    \end{tabular}
\end{table} \\
Thus since $L(\vec{x}+\vec{y})=L(\vec{x})+L(\vec{y})$, the property of additive linearity is satisified.\\
\textbf{\underline{Scalar Linearity:}}\\
Let $\vec{x}\in\mathbb{R}^3$ and $t\in\mathbb{R}$, then we form and simplify as follows,

\begin{table}[htp]
    \centering
    \begin{tabular}{ccll}
        $L(t\vec{x})$ & $=$ & $((t\vec{x})\times\vec{v})+((t\vec{x})\cdot\vec{v})\vec{v}$ & Definition of $L$\\
         & $=$ & $t(\vec{x}\times\vec{v})+(t(\vec{x}\cdot\vec{v}))\vec{v}$ & Associativity of Cross Product and Dot Product\\
         & $=$ & $t(\vec{x}\times\vec{v})+t\vec{v}(\vec{x}\cdot\vec{v})$  & Distributivity\\
         & $=$ & $t(\vec{x}\times\vec{v})+t(\vec{v}(\vec{x}\cdot\vec{v}))$ & Associativity\\
         & $=$ & $t(\vec{x}\times\vec{v})+t((\vec{x}\cdot\vec{v})\vec{v})$ & Commutativity\\
         & $=$ & $t((\vec{x}\times\vec{v})+((\vec{x}\cdot\vec{v})\vec{v}))$& Distributivity \\
         & $=$ & $t\;L(\vec{x})$ & Definition of $L$
    \end{tabular}
\end{table} \\
Thus since $L(t\vec{x})=t\;L(\vec{x})$, the property of scalar linearity is satisfied. \\ \\Since Additive Linearity and Scalar Linearity both hold, $L$ is a linear operator.
\newpage
\subsection*{b)}
To compute $[L]$ we need to determine the result of $L$ on the standard basis vectors of $\mathbb{R}^3$, which are $\vec{e_1},\vec{e_2}$ and $\vec{e_3}$. Thus $[L]$ is given by: $$[L]=\begin{bmatrix}
    | & | & | \\
    L(\vec{e_1}) & L(\vec{e_2}) & L(\vec{e_3})\\
    | & | & |
\end{bmatrix}$$ Thus we perform the following computations:\\

\begin{table}[ht]
    \centering
    \begin{tabular}{l|l|l}
        \;\;\;\;\;\;\;\;\;$\underline{L(\vec{e_1})}$\;\;\;\;\;\;\;\;\;\;\;\;\;\;\;\;\;\;\;\;\;\;\;\; & \;\;\;\;\;\;\;\;\;$\underline{L(\vec{e_2})}$\;\;\;\;\;\;\;\;\;\;\;\;\;\;\;\;\;\;\;\;\;\;\;\; & \;\;\;\;\;\;\;\;\;$\underline{L(\vec{e_3})}$\;\;\;\;\;\;\;\;\;\;\;\;\;\;\;\;\;\;\;\;\;\;\;\;\\
        
         &  & \\
         
        $\underline{\vec{e_1}\times\vec{v}}$: & 
        $\underline{\vec{e_2}\times\vec{v}}$: & 
        $\underline{\vec{e_3}\times\vec{v}}$:\\

         &  & \\
         
       $ =\begin{bmatrix}
            0(v_3) - 0(v_2) \\
            0(v_1) - 1(v_3) \\
            1(v_2) - 0(v_1) 
        \end{bmatrix}=\begin{bmatrix}
            0 - 0 \\
            0 - v_3 \\
            v_2 - 0 
        \end{bmatrix}=\begin{bmatrix}
            0 \\
            - v_3 \\
            v_2  
        \end{bmatrix}$ & 
        $ =\begin{bmatrix}
            1(v_3) - 0(v_2) \\
            0(v_1) - 0(v_3) \\
            0(v_2) - 1(v_1) 
        \end{bmatrix}=\begin{bmatrix}
            v_3 - 0\\
            0 - 0 \\
            0 - v_1 
        \end{bmatrix}=\begin{bmatrix}
            v_3\\
            0 \\
            - v_1 
        \end{bmatrix}$& 
        $ =\begin{bmatrix}
            0(v_3) - 1(v_2) \\
            1(v_1) - 0(v_3) \\
            0(v_2) - 0(v_1) 
        \end{bmatrix}=\begin{bmatrix}
            0- v_2 \\
            v_1 - 0 \\
            0 - 0 
        \end{bmatrix}=\begin{bmatrix}
            - v_2 \\
            v_1  \\
            0
        \end{bmatrix}$\\
        
         &  & \\
         &  & \\
         
        $\underline{\vec{e_1}\cdot \vec{v}}$: & 
        $\underline{\vec{e_2}\cdot \vec{v}}$: & 
        $\underline{\vec{e_3}\cdot \vec{v}}$:\\
        $=1v_1 + 0v_2 + 0v_3=v_1$ & 
        $=0v_1 + 1v_2 + 0v_3=v_2$ & 
        $=0v_1 + 0v_2 + 1v_3=v_3$\\
        
         &  & \\
         &  & \\
         
         $\underline{(\vec{e_1}\cdot \vec{v})\vec{v}}$: & 
         $\underline{(\vec{e_2}\cdot \vec{v})\vec{v}}$: & 
         $\underline{(\vec{e_3}\cdot \vec{v})\vec{v}}$:\\


         
         $=(v_1)\vec{v}=v_1\begin{bmatrix}v_1\\v_2\\v_3\end{bmatrix}
         =\begin{bmatrix}v_1^2\\v_1v_2\\v_1v_3\end{bmatrix}$& 
         
         $=(v_2)\vec{v}=v_2\begin{bmatrix}v_1\\v_2\\v_3\end{bmatrix}
         =\begin{bmatrix}v_2v_1\\v_2^2\\v_2v_3\end{bmatrix}$&
         
         $=(v_3)\vec{v}=v_3\begin{bmatrix}v_1\\v_2\\v_3\end{bmatrix}
         =\begin{bmatrix}v_3v_1\\v_3v_2\\v_3^2\end{bmatrix}$\\
         
         &  & \\
         &  & \\
         
         $\underline{L(\vec{e_1})}$& 
         $\underline{L(\vec{e_2})}$ & 
         $\underline{L(\vec{e_3})}$\\

         
         $=(\vec{e_1}\times \vec{v}) +(\vec{e_1}\cdot\vec{v})\vec{v}$&
         $=(\vec{e_2}\times \vec{v}) +(\vec{e_2}\cdot\vec{v})\vec{v}$&
         $=(\vec{e_3}\times \vec{v}) +(\vec{e_3}\cdot\vec{v})\vec{v}$\\

         
        $=\begin{bmatrix}
            0 \\
            - v_3 \\
            v_2  
        \end{bmatrix}+\begin{bmatrix}v_1^2\\v_1v_2\\v_1v_3\end{bmatrix}$ &
        $=\begin{bmatrix}
            v_3\\
            0 \\
            - v_1 
        \end{bmatrix}+\begin{bmatrix}v_2v_1\\v_2^2\\v_2v_3\end{bmatrix}$& 
        $=\begin{bmatrix}
            - v_2 \\
            v_1  \\
            0
        \end{bmatrix}+\begin{bmatrix}v_3v_1\\v_3v_2\\v_3^2\end{bmatrix}$\\
        &  & \\
        $=\begin{bmatrix}
            v_1^2\\v_1v_2-v_3\\v_1v_3+v_2
        \end{bmatrix}$ &
        $=\begin{bmatrix}
            v_2v_1+v_3\\v_2^2\\v_2v_3-v_1
        \end{bmatrix}$ &
        $=\begin{bmatrix}
            v_3v_1-v_2\\v_3v_2+v_1\\v_3^2
        \end{bmatrix}$\\
         &  & \\
         
    \end{tabular}
\end{table} \\
Since we have computed $L(\vec{e_1}),L(\vec{e_2})$ and $L(\vec{e_3})$, $[L]$ is given by:$$[L]=\begin{bmatrix}
    | & | & | \\
    L(\vec{e_1}) & L(\vec{e_2}) & L(\vec{e_3})\\
    | & | & |
\end{bmatrix}=\begin{bmatrix}
    v_1^2 &  v_2v_1+v_3 & v_3v_1-v_2\\
    v_1v_2-v_3 & v_2^2& v_3v_2+v_1\\
    v_1v_3+v_2 &v_2v_3-v_1 &v_3^2 \\
\end{bmatrix}$$
\newpage
\section*{Question 2:}
Let us first find the individual transformation matrices, then we will multiply them:\\
\textbf{\underline{Reflects about the plane $2x_1-x^2+2x_3$:}}\\
We need to find $[\text{refl}_{\vec{n}}]$ for $\vec{n}=\begin{bmatrix}2\\-1\\2\end{bmatrix}$ which is given by: 

$$[\text{refl}_{\vec{n}}]= \begin{bmatrix}
| & | & | \\
\text{refl}_{\vec{n}}(\vec{e_{1}}) & 
\text{refl}_{\vec{n}}(\vec{e_{2}}) & 
\text{refl}_{\vec{n}}(\vec{e_{3}}) \\
| & | & |
\end{bmatrix}$$Thus we need to compute $\text{refl}_{\vec{n}}$ for the three basis vectors:\\
\underline{\textit{$\text{refl}_{\vec{n}}(\vec{e_1})$:}}\\
\\textit{$\text{proj}_{\vec{n}}(\vec{e_1})$:}\\
$\text{
      proj}_{\vec{n}}(\vec{e_1})=\frac
{
\begin{bmatrix}1\\0\\0\end{bmatrix}\cdot\begin{bmatrix}2\\-1\\2\end{bmatrix}
}{
\begin{bmatrix}2\\-1\\2\end{bmatrix}\cdot\begin{bmatrix}2\\-1\\2\end{bmatrix}
}
\begin{bmatrix}2\\-1\\2\end{bmatrix}
=\frac{2}{4+1+4}\begin{bmatrix}2\\-1\\2\end{bmatrix}\
=\frac{2}{9}\begin{bmatrix}2\\-1\\2\end{bmatrix}
=\begin{bmatrix}4/9\\-2/9\\4/9\end{bmatrix}$

 \\\\
$
\text{refl}_{\vec{n}}(\vec{e_1})=
\begin{bmatrix}1\\0\\0\end{bmatrix}
-2\text{proj}_{\vec{n}}(\vec{e_1})
=\begin{bmatrix}1\\0\\0\end{bmatrix}-2\begin{bmatrix}4/9\\-2/9\\4/9\end{bmatrix}
=\begin{bmatrix}1\\0\\0\end{bmatrix}-\begin{bmatrix}8/9\\-4/9\\8/9\end{bmatrix}
=\begin{bmatrix}1/9\\4/9\\-8/9\end{bmatrix}$




 \\ \underline{\textit{$\text{refl}_{\vec{n}}(\vec{e_2})$:}}\\
\textit{$\text{proj}_{\vec{n}}(\vec{e_2})$:}\\
$\text{
      proj}_{\vec{n}}(\vec{e_2})=\frac
{
\begin{bmatrix}0\\1\\0\end{bmatrix}\cdot\begin{bmatrix}2\\-1\\2\end{bmatrix}
}{
\begin{bmatrix}2\\-1\\2\end{bmatrix}\cdot\begin{bmatrix}2\\-1\\2\end{bmatrix}
}
\begin{bmatrix}2\\-1\\2\end{bmatrix}
=\frac{-1}{4+1+4}\begin{bmatrix}2\\-1\\2\end{bmatrix}\
=\frac{-1}{9}\begin{bmatrix}2\\-1\\2\end{bmatrix}
=\begin{bmatrix}-2/9\\1/9\\-2/9\end{bmatrix}$

 \\\\
$
\text{refl}_{\vec{n}}(\vec{e_2})=
\begin{bmatrix}0\\1\\0\end{bmatrix}
-2\text{proj}_{\vec{n}}(\vec{e_2})
=\begin{bmatrix}0\\1\\0\end{bmatrix}-2\begin{bmatrix}-2/9\\1/9\\-2/9\end{bmatrix}
=\begin{bmatrix}0\\1\\0\end{bmatrix}-\begin{bmatrix}-4/9\\2/9\\-4/9\end{bmatrix}
=\begin{bmatrix}4/9\\7/9\\4/9\end{bmatrix}$

 \\ \underline{\textit{$\text{refl}_{\vec{n}}(\vec{e_3})$:}}\\
\textit{$\text{proj}_{\vec{n}}(\vec{e_3})$:}\\
$\text{
      proj}_{\vec{n}}(\vec{e_3})=\frac
{
\begin{bmatrix}0\\0\\1\end{bmatrix}\cdot\begin{bmatrix}2\\-1\\2\end{bmatrix}
}{
\begin{bmatrix}2\\-1\\2\end{bmatrix}\cdot\begin{bmatrix}2\\-1\\2\end{bmatrix}
}
\begin{bmatrix}2\\-1\\2\end{bmatrix}
=\frac{2}{4+1+4}\begin{bmatrix}2\\-1\\2\end{bmatrix}\
=\frac{2}{9}\begin{bmatrix}2\\-1\\2\end{bmatrix}
=\begin{bmatrix}4/9\\-2/9\\4/9\end{bmatrix}$

 \\\\
$
\text{refl}_{\vec{n}}(\vec{e_3})=
\begin{bmatrix}0\\0\\1\end{bmatrix}
-2\text{proj}_{\vec{n}}(\vec{e_3})
=\begin{bmatrix}0\\0\\1\end{bmatrix}-2\begin{bmatrix}4/9\\-2/9\\4/9\end{bmatrix}
=\begin{bmatrix}0\\0\\1\end{bmatrix}-\begin{bmatrix}8/9\\-4/9\\8/9\end{bmatrix}
=\begin{bmatrix}-8/9\\4/9\\1/9\end{bmatrix}$ \\ \\\\
Then we have:
$$[\text{refl}_{\vec{n}}]= \begin{bmatrix}
| & | & | \\
\text{refl}_{\vec{n}}(\vec{e_{1}}) & 
\text{refl}_{\vec{n}}(\vec{e_{2}}) & 
\text{refl}_{\vec{n}}(\vec{e_{3}}) \\
| & | & |
\end{bmatrix}=\begin{bmatrix}
    1/9 & 4/9 &- 8/9\\
    4/9 & 7/9 & 4/9\\
    -8/9 & 4/9 & 1/9
\end{bmatrix}$$ \\ \\ 
\textbf{\underline{Rotates by an angle of $5\pi / 6$ about the $y$ axis:}}\\
The rotation matrix that rotates a vector about the $y$-axis by an angle of $5\pi/6$ is given by:$$[R_{5\pi/6,y}]=\begin{bmatrix}
\cos(5\pi/6) & 0 & \sin(5\pi/6) \\
0 & 1 & 0 \\
-\sin(5\pi/6) & 0 & \cos (5\pi/6)
\end{bmatrix}=\begin{bmatrix}
     -\sqrt{3}/2&0 & 1/2\\
    0&1 &0 \\
    -1/2&0 & -\sqrt{3}/2\\
\end{bmatrix}$$
\textbf{\underline{Reflects about the $yz$-plane:}}\\
The rotation matrix that reflects a vector about the $yz$-plane is given by:$$[R_{y,z}]=\begin{bmatrix}
    -1& 0& 0\\
    0& 1& 0\\
    0& 0&1 \\
\end{bmatrix}$$
\newpage
After computing all three transformation matrices, we multiply them in reverse order to maintain order through matrix multiplication: $$[L]=([R_{y,z}][R_{5\pi/6,y}])[\text{refl}_{\vec{n}}]$$First: $$[R_{y,z}][R_{5\pi/6,y}]=\begin{bmatrix}
    -1& 0& 0\\
    0& 1& 0\\
    0& 0&1 \\
\end{bmatrix}\begin{bmatrix}
     -\sqrt{3}/2&0 & 1/2\\
    0&1 &0 \\
    -1/2&0 & -\sqrt{3}/2\\\end{bmatrix}$$
$$=\begin{bmatrix}
    (-1)(-\frac{\sqrt{3}}{2}) + (0)(0) + (0)(-\frac{1}{2}) & 
    (-1)(0) + (0)(1) + (0)(0) & 
    (-1)(\frac{1}{2}) + (0)(0) + (0)(-\frac{\sqrt{3}}{2})\\
    (0)(-\frac{\sqrt{3}}{2}) + (1)(0) + (0)(-\frac{1}{2}) & 
    (0)(0) + (1)(1) + (0)(0) & 
    (0)(\frac{1}{2}) + (1)(0) + (0)(-\frac{\sqrt{3}}{2})\\
    (0)(-\frac{\sqrt{3}}{2}) + (0)(0) + (1)(-\frac{1}{2}) & 
    (0)(0) + (0)(1) + (1)(0) & 
    (0)(\frac{1}{2}) + (0)(0) + (1)(-\frac{\sqrt{3}}{2})\\
\end{bmatrix}$$
$$=\begin{bmatrix}
    \frac{\sqrt{3}}{2} +0 + 0 & 
    0 + 0 + 0& 
    -\frac{1}{2}+0+0\\
    0 +0+0& 
    0 + 1 + 0 & 
    0 + 0 + 0\\
    0 + 0 + -\frac{1}{2} & 
    0 + 0 + 0 & 
    0 + 0 + -\frac{\sqrt{3}}{2}\\
\end{bmatrix}=\begin{bmatrix}
    \frac{\sqrt{3}}{2} & 0& -\frac{1}{2}\\
    0& 1& 0\\
    -\frac{1}{2} & 0 & -\frac{\sqrt{3}}{2}\\
\end{bmatrix}$$
\\\\Then we compute: $[L]=([R_{y,z}][R_{5\pi/6,y}])[\text{refl}_{\vec{n}}]$ $$
[L]=([R_{y,z}][R_{5\pi/6,y}])[\text{refl}_{\vec{n}}]
=\begin{bmatrix}
    \frac{\sqrt{3}}{2} & 0& -\frac{1}{2}\\
    0& 1& 0\\
    -\frac{1}{2} & 0 & -\frac{\sqrt{3}}{2}\\
\end{bmatrix}\begin{bmatrix}
    1/9 & 4/9 &- 8/9\\
    4/9 & 7/9 & 4/9\\
    -8/9 & 4/9 & 1/9
\end{bmatrix}$$
$$=\begin{bmatrix}
    (-\frac{\sqrt3}{2})(\frac{1}{9})+(0)(\frac{4}{9})+(-\frac{1}{2})(-\frac{8}{9})&
    (-\frac{\sqrt3}{2})(\frac{4}{9})+(0)(\frac{7}{9})+((-\frac{1}{2})(\frac{4}{9})&
    (-\frac{\sqrt3}{2})(-\frac{8}{9})+(0)(\frac{4}{9})+(-\frac{1}{2})(\frac{1}{9})\\

    (0)(\frac{1}{9})+(1)(\frac{4}{9})+(0)(-\frac{8}{9})&
    (0)(\frac{4}{9})+(1)(\frac{7}{9})+(0)(\frac{4}{9})&
    (0)(-\frac{8}{9})+(1)(\frac{4}{9})+(0)(\frac{1}{9})\\

    (-\frac{1}{2})(\frac{1}{9})+(0)(\frac{4}{9})+(-\frac{\sqrt3}{2})(-\frac{8}{9})&
    (-\frac{1}{2})(\frac{4}{9})+(0)(\frac{7}{9})+(-\frac{\sqrt3}{2})(\frac{4}{9})&
    (-\frac{1}{2})(-\frac{8}{9})+(0)(\frac{4}{9})+(-\frac{\sqrt3}{2})(\frac{1}{9})\\
\end{bmatrix}=$$

$$=\begin{bmatrix}
    -\frac{\sqrt3}{18}+0+\frac{8}{18}&
    -\frac{4\sqrt3}{18}+0+-\frac{4}{18}&
    \frac{8\sqrt3}{18}+0+-\frac{1}{18}\\

   0+\frac{4}{9}+0&
    0+\frac{7}{9}+0&
    0+\frac{4}{9}+0\\

    -\frac{1}{18}+0+\frac{8\sqrt3}{18}&
    -\frac{4}{18}+0+-\frac{4\sqrt3}{18}&
    \frac{8}{18}+0+-\frac{\sqrt3}{18}
\end{bmatrix}=\begin{bmatrix}
    \frac{-\sqrt3+8}{18}&
    \frac{-4\sqrt3-4}{18}&
    \frac{8\sqrt3-1}{18}\\

   \frac{4}{9}&
    \frac{7}{9}&
    \frac{4}{9}\\

    \frac{-1+8\sqrt3}{18}&
    \frac{-4-4\sqrt3}{18}&
    \frac{8-\sqrt3}{18}
\end{bmatrix}$$
Finally we have that: $$[L]=\begin{bmatrix}
    \frac{-\sqrt3+8}{18}&
    \frac{-4\sqrt3-4}{18}&
    \frac{8\sqrt3-1}{18}\\

   4/9&
    7/9&
    4/9\\

    \frac{-1+8\sqrt3}{18}&
    \frac{-4-4\sqrt3}{18}&
    \frac{8-\sqrt3}{18}
\end{bmatrix}$$

\newpage
\section*{Question 3:}
\subsection*{a)}
Let us first find a matrix interpretation for $L$ and row reduce it to RREF: $$\begin{bmatrix}
    x_2+2x_2+3x_3+4x_4\\
    5x_1+6x_2+7x_3+8x_4\\
    11x_1+14x_2+17x_3+20x_4
\end{bmatrix}=\begin{bmatrix}
    1& 2 & 3 & 4 \\
    5& 6 & 7 & 8 \\
    11& 14 & 17 & 20 \\
\end{bmatrix}$$
Then we will row reduce it:
$$\begin{bmatrix}
    1& 2 & 3 & 4 \\
    5& 6 & 7 & 8 \\
    11& 14 & 17 & 20 \\
\end{bmatrix}\begin{aligned}
    R_2 +& -5R_1\\
    &\Rightarrow\\
    R_3 +& -11R_1
\end{aligned}\begin{bmatrix}
    1&2&3&4\\
    0&-4&-8&-12\\
    0&-8&-16&-24
\end{bmatrix}\begin{aligned}
    R_2 &\div -4\\
    &\Rightarrow\\
\end{aligned}\begin{bmatrix}
    1&2&3&4\\
    0&1&2&3\\
    0&-8&-16&-24
\end{bmatrix}\begin{aligned}
    R_3 +& 8R_2\\
    &\Rightarrow\\
\end{aligned}\begin{bmatrix}
    1&2&3&4\\
    0&1&2&3\\
    0&0&0&0
\end{bmatrix}$$


\subsubsection*{i. \textit{$\text{Col}(A)$}}
By observing the row reduced matrix, we see that columns 1 \& 2 contain the leading ones so the columns 1 \& 2 of the original matrix form the basis for the column-space of $L$: $$\left\{\begin{bmatrix}
    1\\5\\1
\end{bmatrix},\begin{bmatrix}
    2\\6\\14
\end{bmatrix}\right\}$$

\subsubsection*{ii. \textit{$\text{Null}(A)$}}
The null-space of $A$ is given by its solution to $A\vec{x}=0$ as follows, we can set our row reduced matrix equal to $\vec{0}$ as it has the same solution as the original:
$$\left[\begin{array}{cccc|c}
    1&2&3&4&0\\
    0&1&2&3&0\\
    0&0&0&0&0
\end{array}\right]
=\begin{cases}
    x_1 +2x_2 +3x_3 +4x_4=0\\
    x_2 +2x_3 +3x_4=0\\
\end{cases}$$

Then for the general solution since there is 2 free variables we represent $x_1$ and $x_2$ in terms of $x_3$ and $x_4$, first we find $x_2$:\\
$x_2=-2x_3-3x_4$\\\\
Then we can substitute $x_2$ in $R_1$ and expand and simplify as follows:\\
$x_1 +2(-2x_3-3x_4) +3x_3 +4x_4=0$\\
$x_1 +-4x_3-6x_4 +3x_3 +4x_4=0$\\
$x_1 +-x_3-2x_4=0$\\
$x_1=x_3+2x_4$\\\\
Now we can represent $x_3=a$ and $x_4=b$ to represent our free variables where $a,b\in\mathbb{R}$: 
$$\begin{cases}
   x_1=a+2b\\
   x_2=-2a-3b\\
   x_3=a\\
   x_4=b
\end{cases}\;\;\;\;\;\;\;\Rightarrow\;\;\;\;\;\;\;\vec{x}=\begin{bmatrix}
    a+2b\\
    -2a-3b\\
    a\\
    b
\end{bmatrix}=\begin{bmatrix}
    a\\-2a\\a\\0
\end{bmatrix}+\begin{bmatrix}
    2b\\-3b\\0\\b
\end{bmatrix}=a\begin{bmatrix}
    1\\-2\\1\\0
\end{bmatrix}+b\begin{bmatrix}
    1\\-3\\0\\1
\end{bmatrix}$$
Since $a$ and $b$ are free variables we can assert that the following is a basis for the null-space of $A$ $$\left\{\begin{bmatrix}
    1\\-2\\1\\0
\end{bmatrix},\begin{bmatrix}
    2\\-3\\0\\1
\end{bmatrix}\right\}$$



\subsubsection*{iii. \textit{$\text{Col}(A^T)$}}
The column-space of the transpose of $A$ also known as the row-space of $A$ is given by the non zero rows of the row reduced matrix: $$\left\{\begin{bmatrix}
    1\\2\\3\\4
\end{bmatrix},\begin{bmatrix}
    0\\1\\2\\3
\end{bmatrix}\right\}$$
\newpage
\subsubsection*{iv. \textit{$\text{Null}(A^T)$}}
To find the left null-space or the null-space of the transpose we must first find the transpose of $A$ $$A^T=\begin{bmatrix}
    1 & 5 & 11\\
    2 & 6 & 14\\
    3 & 7 & 17\\
    4 & 8 & 20
\end{bmatrix}$$
Then we row reduce the transposed matrix: $$\begin{bmatrix}
    1 & 5 & 11\\
    2 & 6 & 14\\
    3 & 7 & 17\\
    4 & 8 & 20
\end{bmatrix}\begin{aligned}
    R_2 +& -2R_1\\
    &\Rightarrow\\
    R_3 +& -3R_1\\
    R_4 +& -4R_1
\end{aligned}\begin{bmatrix}
    1 & 5 & 11\\
    0 & -4 & -8\\
    0 & -8 & -16\\
    0 & -12 & -24
\end{bmatrix}\begin{aligned}
    R_2&(-1/4)\\
    &\Rightarrow\\
    R_3&(-1/8)\\
    R_4&(-1/12)
\end{aligned}\begin{bmatrix}
    1 & 5 & 11\\
    0 & 1 & 2\\
    0 & 1 & 2\\
    0 & 1 & 2
\end{bmatrix}\begin{aligned}
    R_3 +& -2R_2\\
    &\Rightarrow\\
    R_4 +& -3R_2
\end{aligned}\begin{bmatrix}
    1 & 5 & 11\\
    0 & 1 & 2\\
    0 & 0 & 0\\
    0 & 0 & 0
\end{bmatrix}$$
Then we set the matrix equal to $\vec{0}$ and we have that: $$\begin{cases}
    x_1+5x_2+11x_3=0\\
    x_2+2x_3=0
\end{cases}$$
We let $x_3=t$ where $t\in\mathbb{R}$ and the represent $x_1$ and $x_2$ in terms of $t$, first we find $x_2$:\\
$x_2=-2x_3=-2t$\\\\
Then we use the above to make a substitution in $R_1$ and expand and simplify as follows:\\
$x_1+5x_2+11x_3=0$\\
$x_1+5(-2x_3)+11x_3=0$\\
$x_1+-10x_3+11x_3=0$\\
$x_1+x_3=0$\\
$x_1=-x_3=-t$\\\\
Then we have that: $$\begin{cases}
    x_1 = -t\\
    x_2 = -2t\\
    x_3 = t\\
    x_4 = 0
\end{cases}\;\;\;\;\;\;\;\Rightarrow\;\;\;\;\;\;\;\;\begin{bmatrix}
    -t\\-2t\\t\\0
\end{bmatrix}=t\begin{bmatrix}
    -1\\-2\\1\\0
\end{bmatrix}$$\\
Since $t$ is a free variable we can assert that the following is a basis for $\text{Null}(A^T)$:$$\left\{\begin{bmatrix}
    -1\\-2\\1\\0
\end{bmatrix}\right\}$$

Finally we have that$$\text{Col}(A)=\text{Span}\left\{\begin{bmatrix}
    1\\5\\1
\end{bmatrix},\begin{bmatrix}
    2\\6\\14
\end{bmatrix}\right\}\;\;\;\;\;\;\;
\text{Null}(A)=\text{Span} $$ $$\text{Col}(A^T)=\text{Span}\left\{\begin{bmatrix}
    1\\2\\3\\4
\end{bmatrix},\begin{bmatrix}
    0\\1\\2\\3
\end{bmatrix}\right\}\;\;\;\;\;\;\;
\text{Null}(A^T)=\text{Span}\left\{\begin{bmatrix}
    -1\\-2\\1\\0
\end{bmatrix}\right\}$$
\subsection*{b)}
By the rank nullity theorem the sum of the rank of a matrix and the nullity of the same matrix is equal to the number of columns in the respective matrix, thus we have that: 
$$\text{Rank}(A)=2\;\;\;\;\text{Nullity}(A)=2\;\;\;\;\text{Rank}(A)+\text{Nullity}(A)=4$$
Meaning that the matrix $A$ has $4$ columns which is true.\\\\\\
We also have that: 
$$\text{Rank}(A^T)=2\;\;\;\;\text{Nullity}(A^T)=1\;\;\;\;\text{Rank}(A^T)+\text{Nullity}(A^T)=3$$
Meaning that the matrix $A^T$ has 3 columns which is true.\\\\Since the rank is $2$ it means that the function $L$ collapses the input in $\mathbb{R}^4$ to some plane in $\mathbb{R}^3$
\newpage
\section*{Question $4:$}
\subsection*{a)}
To find the LU factorization of $A$ we will first row reduce $A$, while keeping track of the elementary matrices respective to each row operation:$$\begin{bmatrix}
    1&-1&-1\\
    -2&3&1\\
    3&-7&2
\end{bmatrix}\begin{aligned}
    R_2&+2R_1\\
    &\Rightarrow
\end{aligned}\begin{bmatrix}
    1&-1&-1\\
    0&1&-1\\
    3&-7&2
\end{bmatrix}\begin{aligned}
    R_3&-3R_1\\
    &\Rightarrow
\end{aligned}\begin{bmatrix}
    1&-1&-1\\
    0&1&-1\\
    0&-4&5
\end{bmatrix}\begin{aligned}
    R_3&+4R_2\\
    &\Rightarrow
\end{aligned}\begin{bmatrix}
    1&-1&-1\\
    0&1&-1\\
    0&0&1
\end{bmatrix}$$
$$E_1=\begin{bmatrix}
    1 & 0 & 0\\
    2 & 1 & 0\\
    0 & 0 & 1
\end{bmatrix}\;\;\;\;\;\;\;\;\;\;\;\;
E_2=\begin{bmatrix}
    1 & 0 & 0\\
    0 & 1 & 0\\
    -3 & 0 & 1
\end{bmatrix}\;\;\;\;\;\;\;\;\;\;\;\;
E_3=\begin{bmatrix}
    1 & 0 & 0\\
    0 & 1 & 0\\
    0 & 4 & 1
\end{bmatrix}\;\;\;\;\;\;$$
Then we find the respective inverted elementary matrices since $A=(E_1^{-1}E_2^{-1}E_3^{-1})U$ and $L=(E_1^{-1}E_2^{-1}E_3^{-1})$
$$E_1^{-1}=\begin{bmatrix}
    1 & 0 & 0\\
    -2 & 1 & 0\\
    0 & 0 & 1
\end{bmatrix}\;\;\;\;\;\;\;\;\;\;\;\;
E_2^{-1}=\begin{bmatrix}
    1 & 0 & 0\\
    0 & 1 & 0\\
    3 & 0 & 1
\end{bmatrix}\;\;\;\;\;\;\;\;\;\;\;\;
E_3^{-1} =\begin{bmatrix}
    1 & 0 & 0\\
    0 & 1 & 0\\
    0 & -4 & 1
\end{bmatrix}\;\;\;\;\;\;$$
Now to find $L$ we need only multiply $E_1^{-1}(E_2^{-1}E_3^{-1})$ $$L=\begin{bmatrix}
    1 & 0 & 0\\
    -2 & 1 & 0\\
    0 & 0 & 1
\end{bmatrix}\left(\begin{bmatrix}
    1 & 0 & 0\\
    0 & 1 & 0\\
    3 & 0 & 1
\end{bmatrix}\begin{bmatrix}
    1 & 0 & 0\\
    0 & 1 & 0\\
    0 & -4 & 1
\end{bmatrix}\right)=\begin{bmatrix}
    1 & 0 & 0\\
    -2 & 1 & 0\\
    0 & 0 & 1
\end{bmatrix}\left(\begin{bmatrix}
      1 & 0 & 0\\
    0 & 1 & 0\\
    3 & -4 & 1
\end{bmatrix}\right)=\begin{bmatrix}
      1 & 0 & 0\\
    -2 & 1 & 0\\
    3 & -4 & 1
\end{bmatrix}$$
Thus we have that: $$A=LU=\begin{bmatrix}
      1 & 0 & 0\\
    -2 & 1 & 0\\
    3 & -4 & 1
\end{bmatrix}\begin{bmatrix}
    1&-1&-1\\
    0&1&-1\\
    0&0&1
\end{bmatrix}$$
\newpage
\subsection*{b)}
To solve the system $A\vec{x}=\vec{b}$, we will first make the substitution $\vec{y}=U\vec{x}$: $$L\vec{y}=\vec{b}=\begin{bmatrix}
      1 & 0 & 0\\
    -2 & 1 & 0\\
    3 & -4 & 1
\end{bmatrix}\begin{bmatrix}
    y_1\\y_2\\y_3
\end{bmatrix}=\begin{bmatrix}
    1\\-1\\1
\end{bmatrix}\;\;\;\Rightarrow\;\;\;
\begin{cases}
  
    y_1=1\\
 -2(y_1)+y_2=-1\\ 
     3(y_1)-4(y_2)+y_3=1
\end{cases}$$
Then we can use forward-substitution to solve for $\vec{y}$\\
$R_1$: $y_1=1$\\\\$\begin{aligned}
R_2: &-2(y_1)+y_2=-1\\
&-2+y_2=-1\\
&y_2=1\\
\end{aligned}$ \\ \\\\
$\begin{aligned}
    R_3: & 3(y_1)-4(y_2)+y_3=1\\
    & 3(1)-4(1)+y_3=1\\
    & -1+y_3=1\\
    & y_3=2
\end{aligned}$\\\\\\
We now have that: $$\vec{y}=\begin{bmatrix}
    1\\1\\2
\end{bmatrix}$$
Now having $\vec{y}$ we can solve the system $U\vec{x}=\vec{y}$ and the computed $\vec{x}$ will be the solution to the matrix $[A|\vec{x]}]$:
$$U\vec{x}=\vec{y}=\begin{bmatrix}
    1&-1&-1\\
    0&1&-1\\
    0&0&1
\end{bmatrix}\begin{bmatrix}
    x_1\\x_2\\x_3
\end{bmatrix}=\begin{bmatrix}
    1\\1\\2
\end{bmatrix}\;\;\;\Rightarrow\;\;\;\begin{cases}
    x_1-x_2-x_3=1\\
    x_2-x_3=1\\
    x_3=2\\
\end{cases}$$
Then we can use back-substitution to solve for $\vec{x}$\\
$R_3$: $x_3=2$\\
\\$\begin{aligned}
R_2: &x_2-x_3=1\\
&x_2-2=1\\
&x_2=3\\
\end{aligned}$ \\ \\\\
$\begin{aligned}
    R_1: & \\
    & x_1-x_2-x_3=1\\
    & x_1-3-2=1\\
    & x_1=6
\end{aligned}$\\\\\\
We now have that: $$\vec{x}=\begin{bmatrix}
    6\\3\\2
\end{bmatrix}$$
Which is the solution to: $$[A|\vec{b}]=[LU|\vec{b}]=\vec{x}=\begin{bmatrix}
    6\\3\\2
\end{bmatrix}$$
As desired.





\end{document}
