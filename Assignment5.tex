\documentclass{article}
\usepackage[legalpaper, portrait, margin=0.5in]{geometry}
\usepackage{amsmath}
\usepackage{amssymb}
\usepackage{cancel}

\title{Math 211 Assignment 5}
\author{Dryden Bryson}
\date{Thursday 28th of November}

\begin{document}

\maketitle
\newpage
\section*{Question 1:}
We will first start by performing row operations to reduce row 1 to a row containing only $x$:
$$\begin{vmatrix}
x&3&4\\
x^{2}&9&16\\
7&x+4&x+3
\end{vmatrix}\;\;\begin{aligned}
    &\rightarrow\\
    C_{3}&-\frac{4}{3}C_{2}
\end{aligned}\;\;\begin{vmatrix}
x&3&0\\
x^{2}&9&4\\
7&x+4&\frac{-x-7}{3}
\end{vmatrix}\;\;\begin{aligned}
    &\rightarrow\\
    C_{2}&-\frac{3}{x}C_{1}
\end{aligned}\;\;\begin{vmatrix}
    x&0&0\\
    x^{2}&9-3x&4\\
    7&x+4-\frac{21}{x}&\frac{-x-7}{3}
    \end{vmatrix}
$$
Thus by the cofactor expansion formula we have that:
$$p(x)=x\cdot \begin{vmatrix}
9-3x&4\\
x+4-\frac{21}{x}&\frac{-x-7}{3}
\end{vmatrix}$$
Then by the formula for the determinant of a $2\times 2$ matrix we have that:
\begin{table}[htp]
\centering
\begin{tabular}{cccclc}
  $p(x)$ & $=$ & $x\cdot \begin{vmatrix}9-3x&4\\x+4-\frac{21}{x}&\frac{-x-7}{3} \end{vmatrix}$ & $=$ & $x\cdot ((9-3x)(\frac{-x-7}{3})-4(x+4-\frac{21}{x}))$  &   \\
        && & $=$ & $x\cdot (\frac{(9-3x)(-x-7)}{3}-(4x+16-\frac{84}{x}))$  &   \\
        && & $=$ & $x\cdot (\frac{3x^{2}+12x-63}{3}-(4x+16-\frac{84}{x}))$  &   \\
        && & $=$ & $x\cdot ((x^{2}+4x-21)-(4x+16-\frac{84}{x}))$  &   \\
        && & $=$ & $x\cdot (x^{2}+4x-21-4x-16+\frac{84}{x})$  &   \\
        && & $=$ & $x\cdot (x^{2}-37+\frac{84}{x})$  &   \\
        && & $=$ & $x^{3}-37x+84x$  &   \\
\end{tabular}
\end{table}\\
Now it suffices to find the roots of: $$p(x)=x^{3}-37x+84x$$
Through the use of the rational root theorem we can identify $3$ as one of the roots, thus we can perform synthetic division to form a quadratic which we can solve using the quadratic formula, we need to perform: $$(x^{3}-37x+84) \div (x-3)$$
The following synthetic division table: 
$$\begin{array}{r|rrrr}
3 & 1 & 0 & -37 & 84 \\
  &   & 3 & 9    & -84 \\
\hline
  & 1 & 3 & -28  & 0 \\
\end{array}$$
We have that the factor expression is: $$(x-3)(x^{2}+3x-28)$$
By using the quadratic formula on $(x^{2}+3x-28)$ we have that the remaining roots are, $4$ and $-7$ which gives us: $$p(x)=\begin{vmatrix}
    x&3&4\\
    x^{2}&9&16\\
    7&x+4&x+3
    \end{vmatrix}=x^{3}-37x+84x=(x-3)(x-4)(x+7)$$
    Thus we have that the roots of $p(x)$ are $3$,$4$ and $-7$

\newpage
\section*{Question 2:}
\subsection*{a)}
Since we know that if two matrices have the same characteristic polynomial then they have the same eigen values. Thus it suffices to show that $A$ and $A^{T}$ have the same characteristic polynomial. The characteristic polynomial of $A^{T}$is $(A^{T}-\lambda I)$\\\\ 
We will first demonstrate that: $(\lambda I)^{T}=(\lambda I)$ we can observe this visually: 
$$(\lambda I) = 
\begin{bmatrix} 
\lambda & 0 & 0 & \cdots & 0 \\ 
0 & \lambda & 0 & \cdots & 0 \\ 
0 & 0 & \lambda & \cdots & 0 \\ 
\vdots & \vdots & \vdots & \ddots & \vdots \\ 
0 & 0 & 0 & \cdots & \lambda 
\end{bmatrix}\;\;\;\;\;\;\;(\lambda I)^T = 
\begin{bmatrix} 
\lambda & 0 & 0 & \cdots & 0 \\ 
0 & \lambda & 0 & \cdots & 0 \\ 
0 & 0 & \lambda & \cdots & 0 \\ 
\vdots & \vdots & \vdots & \ddots & \vdots \\ 
0 & 0 & 0 & \cdots & \lambda 
\end{bmatrix}$$
Which makes sense since the matrices are diagonally symmtrical, thus clearly: $(\lambda I)^{T}=(\lambda I)$. We can proceed with the property of "Determinant of Transpose" where: $$\left\vert A^{T} \right\vert = \left\vert A \right\vert $$
Thus we have that $$\left\vert A-\lambda I \right\vert = \left\vert (A-\lambda I )^{T} \right\vert $$
And then by the transpose property of matrix addition we have that: $$\left\vert A-\lambda I \right\vert = \left\vert A^{T}-(\lambda I)^{T}  \right\vert $$
and since we have shown that $(\lambda I)^{T}=(\lambda I)$ we have that: $$\left\vert A-\lambda I \right\vert = \left\vert A^{T}-\lambda I  \right\vert $$
Which demonstrates that the characteristic polynomial of a matrix is the same as the characteristic polynomial for the transposed matrix, thus the matrices $A$ and $A^{T}$ will have the same eigen values. $\square$
\subsubsection*{b)}
We have that $A\vec{v}=\lambda \vec{v}$ and we want to show that $(A-kI)\vec{v}=(\lambda-k)\vec{v}$,  thus we will being by observing the matrix $A-kI$ acting on $\vec{v}$:
\begin{table}[htp]
\centering
\begin{tabular}{ccll}
  $(A-kI)\vec{v}$ & $=$  & $A\vec{v}-kI\vec{v}$        & Right Distributivity of Matrix Multiplication \\
                  & $=$  & $A\vec{v}-k\vec{v}$         & Multiplication of identity matrix ($IM=M$)\\
                  & $=$  & $\lambda\vec{v}-k\vec{v}$   & Substitution from given information ($A\vec{v}=\lambda\vec{v}$)  \\
                  & $=$  & $(\lambda-k)\vec{v}$        & Right Distributivity of Matrix Multiplication  \\
\end{tabular}
\end{table}\\
We have shown that: $(A-kI)\vec{v}=(\lambda-k)\vec{v}$ or that $\vec{v}$ is an eigen vector of the matrix $A-kI$ with and eigen value of $\lambda -k$. $\square$
\subsubsection*{c)}
Since $A$ is idempotent we have that: $$A=A^{2}\Rightarrow A\vec{v}=A^{2}\vec{v}$$
Next we can represent $A^{2}$ as follows: 
\begin{table}[htp]
\centering
\begin{tabular}{ccll}
  $A^{2}\vec{v}$ & $=$ &  $A(A\vec{v})$ & Definition of exponentiation  \\
                 & $=$ &  $A(\lambda \vec{v})$ & Subsitution from eigenvalue equation $A\vec{v}=\lambda\vec{v}$  \\
                 & $=$ &  $\lambda(A\vec{v})$  & Associativity of Scalars  \\
                 & $=$ &  $\lambda(\lambda\vec{v})$ &   Subsitution from eigenvalue equation $A\vec{v}=\lambda\vec{v}$\\
                 & $=$ &  $\lambda^{2}\vec{v}$ &  Definition of exponentiation\\
\end{tabular}
\end{table}\\
We can connect the two equations which yeilds: $$A\vec{v}=A^{2}\vec{v}=\lambda^{2}\vec{v}$$
Then we can form and simplify the following eigenvalue equation: 
\begin{table}[htp]
\centering
\begin{tabular}{rcll}
 $A\vec{v}$        & $=$ & $\lambda^{2}\vec{v}$ & From above equation  \\
 $\lambda \vec{v}$ & $=$ & $\lambda^{2} \vec{v}$  & Substitution from eigenvalue equation  \\
 $\lambda \vec{v} - \lambda^{2} \vec{v}$            & $=$ & $0$  &  Subtract $\lambda^{2}\vec{v}$ from both sides\\
 $(\lambda  - \lambda^{2}) \vec{v}$            & $=$ & $0$  &  Distributivity of Scalar Addition \\
 $(\lambda  - \lambda^{2})$            & $=$ & $0$  &  Divide both sides by $\vec{v}$ \\
 $\lambda(1  - \lambda)$            & $=$ & $0$  &  Distributivity \\
\end{tabular}
\end{table}\\
Then trivially we can see how the only possible values for $\lambda$ are 0 and 1, since $\lambda(1-\lambda)=0$ thus $\lambda_{1}=0, \lambda_{2}=1$. $\square$

\newpage
\section*{Question 3:}
\subsection*{a)}
We will begin with the cofactor decomposition of $A-\lambda I$ along the first row, which is as follows:
$$\begin{vmatrix}
1-\lambda & 3 & 3\\
-3 & -5-\lambda & -3\\
3 & 3& 1-\lambda
\end{vmatrix}=1-\lambda\begin{vmatrix}
-5-\lambda& -3\\
3 & 1-\lambda
\end{vmatrix}-3\begin{vmatrix}
-3 & -3 \\
3 & 1-\lambda
\end{vmatrix}+3\begin{vmatrix}
-3 & -5-\lambda \\
3 & 3
\end{vmatrix}$$
Let us then compute the sub-determinants:\\ 
i) $\begin{aligned}
    \begin{vmatrix}
        -5-\lambda& -3\\
        3 & 1-\lambda
        \end{vmatrix} &= (1-\lambda)(-5-\lambda)-(3\cdot -3)\\
        &=\lambda^{2}+4\lambda-5+9\\
        &=\lambda^{2}+4\lambda+4\\
        &=(\lambda+2)(\lambda+2)   
\end{aligned}$\\\\\\ii) $\begin{aligned}
    \begin{vmatrix}
        -3 & -3 \\
        3 & 1-\lambda
        \end{vmatrix} &= -3(1-\lambda)-(3\cdot -3)\\
        &= -3+3\lambda+9\\
        &=3\lambda+6
\end{aligned}$\\\\\\iii) $\begin{aligned}
    \begin{vmatrix}
        -3 & -5-\lambda \\
        3 & 3
        \end{vmatrix} &= (3\cdot -3)-3(-5-\lambda)\\
        &=-9+15+3\lambda\\
        &=3\lambda+6
\end{aligned}$\\\\
Which gives us the entire equation: 
$$\begin{vmatrix}
1-\lambda & 3 & 3\\
-3 & -5-\lambda & -3\\
3 & 3& 1-\lambda
\end{vmatrix}=1-\lambda(\lambda+2)(\lambda+2) \cancel{-3(3\lambda+6)}\cancel{+3(3\lambda+6)}$$
Thus we have that the simplified characteristic polynomial is: $$(1-\lambda)(\lambda+2)(\lambda+2)$$
Which gives us the following eigen values: $\lambda_{1}=\lambda_{2}=-2,\lambda_{3}=1$
\newpage
\subsection*{b)}
We need to setup the homogenous equations $(A-\lambda I)\vec{v}=0$ for each $\lambda$
\subsubsection*{Case i) $\lambda=-2$}
We have that: $$A-(-2)I=\begin{bmatrix} 
3&3&3\\
-3&-3&-3\\
3&3&3
\end{bmatrix}$$ Thus we form the homogenous system and reduce it: $$\left[\begin{array}{ccc|c}
    3&3&3&0\\
    -3&-3&-3&0\\
    3&3&3&0
\end{array}\right]\begin{aligned}
    R_{2}&+R_{1}\\
    R_{3}&-R_{1}\\
    &\rightarrow
\end{aligned}\left[\begin{array}{ccc|c}
3&3&3&0\\
0&0&0&0\\
0&0&0&0\\
\end{array}\right]\begin{aligned}
    R_{1}&\cdot \frac{1}{3}\\
    &\rightarrow
\end{aligned}\left[\begin{array}{ccc|c}
    1&1&1&0\\
    0&0&0&0\\
    0&0&0&0\\
    \end{array}\right]$$
    Thus we have that $v_{1}+v_{2}+v_{3}=0$, let $v_{2}=s, v_{3}=t$ thus we have the system: $$\begin{cases}
    v_{1}=-s-t\\
    v_{2}=s\\
    v_{3}=t
    \end{cases}$$
    And so: $$
    \vec{v}=\begin{bmatrix} 
    -s-t\\s\\t
    \end{bmatrix}=s\begin{bmatrix} 
    -1\\1\\0
    \end{bmatrix}+t\begin{bmatrix} 
    -1\\0\\1
    \end{bmatrix}\;\;\;\;\;\;\;\;\;\text{thus:}\;\;\;\;\;\vec{v_{1}}=\begin{bmatrix} 
    -1\\1\\0
    \end{bmatrix},\;\;\vec{v_{2}}=\begin{bmatrix} 
    -1\\0\\1
    \end{bmatrix}$$
\subsubsection*{Case ii) $\lambda = 1$}
We have that: $$A-(1)I=\begin{bmatrix}
0&3&3\\
-3&-6&-3\\
3&3&0
\end{bmatrix}$$
Thus we form the homogenous system and reduce it: $$
\left[\begin{array}{ccc|c}
        0&3&3 & 0\\
    -3&-6&-3 & 0\\
    3&3&0 & 0
    \end{array}\right]\begin{aligned}
        R_{2}&+R_{1}\\
        &\rightarrow\\
        R_{2}&+R_{3}
    \end{aligned}\left[\begin{array}{ccc|c}
        0 & 3 & 3 & 0\\
    0 & 0 & 0 & 0\\
    3 & 3& 0 & 0
    \end{array}\right]\begin{aligned}
        R_{3}&-R_{1}\\
        &\rightarrow
    \end{aligned}\left[\begin{array}{ccc|c}
        0 & 3 & 3 & 0\\
    0 & 0 & 0 & 0\\
    3 & 0 & -3 & 0\\
    \end{array}\right]\begin{aligned}
        R_{1} &\leftrightarrow R_{2}\\
        &\rightarrow\\
        R_{3} &\leftrightarrow R_{1}
    \end{aligned}\left[\begin{array}{ccc|c}
    3 & 0 & -3 & 0\\
    0 & 3 & 3 & 0\\
    0 & 0 & 0 & 0
    \end{array}\right]\begin{aligned}
        R_{1}& \cdot 1/3\\
        & \rightarrow\\
        R_{2} &\cdot 1/3
    \end{aligned}\left[\begin{array}{ccc|c}
    1 & 0 & -1 & 0\\
    0 & 1 & 1 & 0\\
    0 & 0 & 0 & 0
    \end{array}\right]$$
Thus we have that: $v_{1}-v_{3}=0$ and $v_{2}+v_{3}=0$. Let $v_{3}=t$ thus we have the system: $$\begin{cases}
v_{1}=t\\
v_{2}=-t\\
v_{3}=t
\end{cases}$$ And so:
$$\vec{v}=\begin{bmatrix} 
t\\-t\\t
\end{bmatrix}=t\begin{bmatrix} 
1\\-1\\1
\end{bmatrix}\;\;\;\;\;\;\;\;\;\text{thus:}\;\;\;\;\;v_{3}=\begin{bmatrix} 
1\\-1\\1
\end{bmatrix}$$
Thus we have the following eigenpairs: $$
\lambda_{1}=\lambda_{2}=-2,\vec{v_{1}}=\begin{bmatrix} 
-1\\1\\0
\end{bmatrix},\vec{v_{2}}=\begin{bmatrix} 
-1\\0\\1
\end{bmatrix}\;\;\text{and}\;\;\lambda_{3}=1,\vec{v_3}=\begin{bmatrix} 
1\\-1\\1
\end{bmatrix}$$\\\\\\\\\\\\\\\\\\\\\\\\\\\\\\
\subsubsection*{\textbf{\underline{$A$ is Diagonalizable:}}}
Per the diagonalization theorem, our $3\times 3$ matrix $A$ can be diagonalized if and only if our eigenvectors form a basis for $\mathbb{R}^{3}$. 
We know they form a basis if the system is not defficient and since our matrix has a geometric and algebreaic multiplicity of $3$, $A$ is not defficient 
and therefore the eigenvectors are linearly independent and form a basis for $\mathbb{R}^{3}$. As a result the matrix is diagonalizable. 

\newpage
\subsection*{c)}
We aim to construct $A=PDP^{-1}$ and we have that: $$D=\begin{bmatrix} 
-2 & 0 & 0\\
0 & -2 & 0\\
0 & 0 & -2\\
\end{bmatrix}\;\;\;\;\;\;\;\;P=\begin{bmatrix} 
-1 & -1 & 1\\
1 & 0 & -1\\
0 & 1 & 1
\end{bmatrix}$$
and thus it suffice to find $P^{-1}$ in order to find the diagonalization $A=PDP^{-1}$. 
To compute $P^{-1}$ we set up the augmented matrix $[P|I]$ and reduce as follows: $$\left[\begin{array}{ccc|ccc}
-1&-1&1      &      1&0&0\\
1&0&-1       &      0&1&0\\
0&1&1        &      0&0&1
\end{array}\right]$$
    

\end{document}